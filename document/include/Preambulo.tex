% ---  GENERALIDADES DEL DOCUMENTO  ---
\documentclass[12pt,openany]{book}
\usepackage{lipsum} % Genera Texto de Relleno
\usepackage[utf8]{inputenc}
\usepackage[T1]{fontenc}
\usepackage[english,spanish,es-lcroman]{babel}
\usepackage{bookman} % Fuente "Bookman" para todo el documento
\usepackage[letterpaper, top=3cm, bottom=3cm, right=2.54cm, left=2.54cm]{geometry}
\usepackage{afterpage}
\usepackage[
  pdftex,
  pdfauthor={\NombreAutor{}},
  pdftitle={Trabajo de Fin de Grado},
  pdfborder={0 0 0}
]{hyperref}
\hypersetup{
    % bookmarks       = true,             % Show the bookmarks
	unicode         = true,             % Use Unicode                 
	pdftoolbar      = true,             % Show Acrobat’s toolbar
	pdfmenubar      = true,             % Show Acrobat’s menu
	pdffitwindow    = false,            % Window fit to page when opened
	pdfstartview    = {FitH},           % Fit the page width to the window
	pdfnewwindow    = true,             % Open the links in a new PDF window
	colorlinks      = true,             % Use colored links
	linkcolor       = blue,             % Internal links color
	citecolor       = green,            % Bibliographic links color
	filecolor       = cyan,             % File links color
	urlcolor        = cyan           % External links color
}
\usepackage{pdfpages} % Permite cargar pdf ajenos a Latex
\usepackage{url}
\usepackage[stable]{footmisc} % Para pie de páginas
\usepackage{parskip} % para separar párrafos con espacio.
\usepackage{lscape} % Permite rotar el contenido y la pagina
\usepackage{pdflscape} % Permite de forma fisica la pagina al exportar
\usepackage{rotating}
\usepackage{pgfgantt} % Permite crear diagramas de Gannt
% ---  GENERALIDADES DEL DOCUMENTO  ---


% --- ECUACONES, IMAGES, FIGURAS, TABLAS  ---
\usepackage{amsfonts,amsgen,amsmath,amssymb} 
\decimalpoint
\usepackage{graphicx} % Permite el uso de imagenes
% \usepackage{caption}
\addto\captionsspanish{\renewcommand{\tablename}{Tabla}} % Cambiar el nombre para las tablas en español
% \usepackage{subcaption}
\usepackage{float}
\usepackage{subfigure}
\usepackage{booktabs}
\usepackage{colortbl,longtable} %Tablas
% --- ECUACONES, IMAGES, FIGURAS, TABLAS  ---


% --- BIBLIOGRAPHY ---
\usepackage[
  backend=biber,
  style=ieee,
  sorting=none
]{biblatex}
\addbibresource{document/include/bibliography.bib}
% \title{Gestión de bibliografía: paquete \texttt{biblatex}}
\usepackage{csquotes} % Insertar frases o citas
% --- BIBLIOGRAPHY ---


% --- GLOSARIO, ACRÓNIMOS (ABREVITURAS), NOMENCLATURA ---
% \usepackage[toc, acronym]{glossaries} 
% \makeglossaries
% %  % Glosario (Palabras con su definición)

% \newglossaryentry{hab}
% {
%     name=HAB,
%     description={Globo de Gran Altitud por sus siglas en inglés de “High-altitude balloon”. También son llamados como Globo estratosférico, es un globo no tripulado lleno de helio o hidrogeno que se libera y es capaz de llegar a la estratosfera.}
% }

% \newglossaryentry{sonda}{
%     name=sonda,
%     description={comprende todos los componentes de una misión HAB como globo, carga, útil, paracaídas, sistemas electrónicos y mecánicos si hubiese.}
% }

% \newglossaryentry{lru}{
%     name=LRU,
%     description={ Line Replaceable Unit, es un componente modular de un vehículo fabricado que puede ser reemplazado fácilmente en caso de que este falle.}
% }

% \newglossaryentry{isa}{
%     name=ISA,
%     description={International Standard Atmosphere o Estándar Atmosférico Internacional, es un modelo atmosférico que permite obtener valores estándares de presión, temperatura, densidad y viscosidad del aire en función de la altitud.}
% }

% % Acrónimos 
% \newacronym{omm}{OMM}{Observatorio Micro Macro, dependencia de la Universidad Don Bosco}

% \newacronym{pcb}{PCB}{circuito impreso, por sus siglas en inglés de “print circuit board”.}

% \newacronym{eda}{EDA}{análisis exploratorio de los datos, EDA por sus siglas en inglés de “Exploratory Data Analysis”.}

% \newacronym{gnss}{GNSS}{Sistema global de navegación por satélite, GNSS, por sus siglas en inglés de “Global Navigation Satellite System”.}

% \newacronym{imu}{IMU}{Unidad de medición inercial, IMU por sus siglas en inglés de “Inertial measurement unit”.}






%  \loadglsentries{document/include/glosario} % cargar las entradas del glosario
% \usepackage[printonlyused]{acronym} % Nomenclatura
% --- GLOSARIO, ACRÓNIMOS (ABREVITURAS), NOMENCLATURA ---


% --- ENCABEZADOS, PIES DE PÁGINA ---
\usepackage{fancyhdr}
\pagestyle{fancy}
\fancyhf{}
\fancyhead[LO]{\leftmark}
\fancyhead[RE]{\rightmark}
\setlength{\headheight}{1.5\headheight}
\cfoot{\thepage}

% --- ENCABEZADOS, PIES DE PÁGINA ---

% --- INDICE GENERAL  ---
\addto\captionsspanish{ \renewcommand{\contentsname}
  {Índice general} }
\setcounter{tocdepth}{4}
\setcounter{secnumdepth}{4}
% --- INDICE GENERAL  ---


%% Borrar si no funciona 
\usepackage{rotating}

% --- ENCABEZADOS DE CAPITULOS Y SECCIONES  ---
\renewcommand{\chaptermark}[1]{\markboth{\textbf{#1}}{}}
\renewcommand{\sectionmark}[1]{\markright{\textbf{\thesection. #1}}}
\newcommand{\HRule}{\rule{\linewidth}{0.5mm}}
\newcommand{\bigrule}{\titlerule[0.5mm]}
% --- ENCABEZADOS DE CAPITULOS Y SECCIONES  ---


% --- ANEXOS  ---
\usepackage{appendix}
\renewcommand{\appendixname}{Anexos}
\renewcommand{\appendixtocname}{Anexos}
\renewcommand{\appendixpagename}{}
% --- ANEXOS  ---


%%-----------------------------------------------
%% Páginas en blanco sin cabecera:
%%-----------------------------------------------
\usepackage{dcolumn}
\newcolumntype{.}{D{.}{\esperiod}{-1}}
\makeatletter
\addto\shorthandsspanish{\let\esperiod\es@period@code}

\def\clearpage{
  \ifvmode
  \ifnum \@dbltopnum =\m@ne
  \ifdim \pagetotal <\topskip
  \hbox{}
  \fi
  \fi
  \fi
  \newpage
  \thispagestyle{empty}
  \write\m@ne{}
  \vbox{}
  \penalty -\@Mi
}
\makeatother




%%-----------------------------------------------
%% Estilos código de lenguajes: Consola, C, C++ y Python
%%-----------------------------------------------
\usepackage{color}

\definecolor{gray97}{gray}{.97}
\definecolor{gray75}{gray}{.75}
\definecolor{gray45}{gray}{.45}

\usepackage{listings}
\lstset{ frame=Ltb,
  framerule=0pt,
  aboveskip=0.5cm,
  framextopmargin=3pt,
  framexbottommargin=3pt,
  framexleftmargin=0.4cm,
  framesep=0pt,
  rulesep=.0pt,
  backgroundcolor=\color{gray97},
  rulesepcolor=\color{black},
  %
  stringstyle=\ttfamily,
  showstringspaces = false,
  basicstyle=\scriptsize\ttfamily,
  commentstyle=\color{gray45},
  keywordstyle=\bfseries,
  %
  numbers=left,
  numbersep=6pt,
  numberstyle=\tiny,
  numberfirstline = false,
  breaklines=true,
}

% Puedes añadir más lenguajes siguiendo un estilo similar al de Python,
% o hacerte tu propio estilo.
% Enlace con alguna ayudita: www.overleaf.com/learn/latex/Code_listing
\lstnewenvironment{listing}[1][]
                  {\lstset{#1}\pagebreak[0]}{\pagebreak[0]}
                    \lstdefinestyle{consola}{
                                    basicstyle=\scriptsize\bf\ttfamily,
                                    backgroundcolor=\color{gray97}}
                    \lstdefinestyle{C}{
                                    basicstyle=\scriptsize,
                                    frame=single,
                                    language=C,
                                    numbers=left}
                    \lstdefinestyle{C++}{
                                    basicstyle=\small,
                                    frame=single,
                                    backgroundcolor=\color{gray75},
                                    language=C++,
                                    numbers=left}
                    \lstdefinestyle{Python}{
                                    basicstyle=\small,
                                    frame=single,
                                    backgroundcolor=\color{gray75},
                                    language=Python,
                                    numbers=left}
                                \makeatother
