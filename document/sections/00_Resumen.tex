\chapter*{Resumen} \label{chp:00_abstract}

El presente trabajo fue desarrollado como una parte más que complementa al proyecto de “Diseño de la misión para un globo meteorológico de gran altitud - StratoBalloon” del Observatorio Micro-Macro (OMM), dependencia de la Universidad Don Bosco. Dentro de ese contexto, se desarrolla una simulación computacional de trayectoria de un globo-sonda a lo largo de su misión para desarrollar un dimensionamiento a un subsistema de navegación.  

\vspace{0.4cm}

El presente se ha dividido en 4 capítulos.

\begin{itemize}
    \item \textbf{CAPÍTULO 1:} Se contextualiza sobre el trabajo de graduación; así como una breve descripción de la misión de StratoBallon.
    \item \textbf{CAPÍTULO 2:} Se desarrolla una simulación de trayectoria con diferentes modelos,  considerando los parámetros más clave de la misión. 
    \item \textbf{CAPÍTULO 3:} Se analiza los datos generados en el simulador para obtener medidas de tendencia central (media, moda, etc.) además de observar el comportamiento de la trayectoria de forma gráfica, con lo anterior se desarrolla un dimensionamiento de un subsistema de navegación en el próximo capítulo. 
    \item \textbf{CAPÍTULO 4:} Se propone un dimensionamiento para un subsistema de navegación apto para la misión del globo meteorológico.

\end{itemize}

\vspace{1.0cm}

\textbf{Palabras Clave:}  Globo de gran altitud, High-altitude balloon, HAB, ISA, trayectoria, modelo, dinámica, sonda, Near-Space, simulación, ascenso, descenso, python, atmósfera, PCB, viento.


%%%%%%%%%%%%%%%%%%%%%%%%%%%%%%%%%%%%%%%%%%%%%%%%%%%%%%%%%%%%%%%%%%%%%%%%%%%%%%%%

\newpage

%%%%%%%%%%%%%%%%%%%%%%%%%%%%%%%%%%%%%%%%%%%%%%%%%%%%%%%%%%%%%%%%%%%%%%%%%%%%%%%%

\chapter*{Abstract}

The present work was developed as a complementary part of the project 'Mission design for a high altitude weather balloon, StratoBalloon' of the Micro-Macro Observatory (MMO) of the Don Bosco University. In this context, a computational simulation of the trajectory of a balloon-sonde along its mission is developed to provide a dimensioning of a navigation subsystem.

\vspace{0.5cm}

This paper is divided into 4 chapters.

\textbf{CHAPTER 1: }Contextualizes the graduation work; as well as a brief description of StratoBallon's mission.

\textbf{CHAPTER 2}: A trajectory simulation is developed with different models, considering the most key parameters of the mission. 

\textbf{CHAPTER 3:} The data generated in the simulator is analyzed to obtain measures of central tendency (mean, mode, etc.) in addition to observing the behavior of the trajectory graphically, with the above to develop a dimensioning of a navigation subsystem in the next chapter.

\textbf{CHAPTER 4: }A dimensioning for a navigation subsystem suitable for the weather balloon mission is proposed.

\vspace{1.5cm}

\textbf{Keywords:} high altitude balloon, High-altitude balloon, HAB, ISA, trajectory, model, dynamics, probe, Near-Space, simulation, ascent, descent, python, atmosphere, PCB, wind.